\documentclass{curs}

% Comment out lines below in case of no code to be included.
%\usepackage{code/highlight}
\usepackage{color}
\usepackage{graphicx}
\usepackage{listings}
\usepackage{multicol}
%\usepackage{alltt}


\title[Session 07]{Session 07}
\subtitle{Application Confinement}
\author{Security of Information Systems (SIS)}
\date{November 9, 2018}

\begin{document}

\frame{\titlepage}

\begin{frame}{Story So Far}
  \begin{itemize}
    \pause \item systems and system components have an attack surface
    \pause \item flaws in systems and system components may be exploited
    \pause \item input may be used maliciously
    \pause \item prevent existance and prevent exploitation of vulnerabilities
    \pause \item defender needs to limit damage
  \end{itemize}
\end{frame}

\begin{frame}{Limiting Damage}
  \begin{itemize}
    \pause \item isolate entire system, e.g. virtualization
    \pause \item isolate/confine system component (application), e.g. sandboxing
    \pause \item limit possible actions, limit accessible resources, e.g. prevent an app from using the network, prevent an app from reading data from other apps
  \end{itemize}
\end{frame}

\begin{frame}{Application Confinement}
  \begin{itemize}
    \pause \item What can an application do? What can an application access?
    \pause \item access control: subject, object
    \pause \item typically enforced at kernel level
    \pause \item What if it were enforced by a library at application level?
    \pause \item overhead
    \pause \item filesystem: users, file permissions, access control lists
    \pause \item configurable permissions: Android permissions, iOS Privacy Settings, Linux capabilities
    \pause \item sandboxing: jailing (filesystem), application sandboxing (kernel-enforced rules)
  \end{itemize}
\end{frame}

\begin{frame}{Remember: Malware}
  \begin{itemize}
    \pause \item application deployed on user device/workstation
    \pause \item may abuse resource use and access
    \pause \item doesn't require a vulnerability in an app, only a defect in the configuration or system
    \pause \item confining it reduces damage
  \end{itemize}
\end{frame}

\begin{frame}{Filesystem Access Control}
  \begin{itemize}
    \pause \item subject: process (UID)
    \pause \item object: file (UID, GID)
    \pause \item permissions or access control lists (attached to a file)
  \end{itemize}
\end{frame}

\begin{frame}{Android Permissions}
  \begin{itemize}
    \pause \item requests permissions at runtime
    \pause \item permission approval
    \pause \item enforcement at Android SDK level
    \pause \item signed permissions
  \end{itemize}
\end{frame}

\begin{frame}{iOS Privacy Settings}
  \begin{itemize}
    \item database mappping between app and resource/service
    \item \texttt{Preferences} app writes to database
    \item may be turned on/off
  \end{itemize}
\end{frame}

\begin{frame}{Linux Capabilities}
  \begin{itemize}
    \item security tokens providing privileges
    \item attached to a given process
    \item allow different permissions for processes belonging to the same user
    \item may also be attached to an executable (similar to the setuid bit)
  \end{itemize}
\end{frame}


\section{Confinement and Enforcement in the Linux Kernel}

\begin{frame}{Linux Security Modules}
  \begin{itemize}
    \item framework in Linux kernel
    \item hooks for user-level system call
    \item introduced in Linux kernel 2.6
  \end{itemize}
\end{frame}

\begin{frame}{MAC Implementations}
  \begin{itemize}
    \item SELinux (2.6.0)
    \item AppArmor (2.6.36)
    \item Smack (2.6.25)
    \item TOMOYO (2.6.30)
    \item Yama (3.4)
  \end{itemize}
\end{frame}

\begin{frame}{SELinux}
  \begin{itemize}
    \item inode based
    \item uses labels - user:role:type:mls
    \item policy based
    \item modes
      \begin{itemize}
        \item disabled
        \item permissive
        \item enforcing
      \end{itemize}
    \item other features
      \begin{itemize}
        \item Role-Based Access Control (RBAC)
        \item Multi-Level Security (MLS)
        \item Multi-Category Security (MCS)
      \end{itemize}
  \end{itemize}
\end{frame}

\begin{frame}{AppArmor}
  \begin{itemize}
    \item path based
    \item filesystem agnostic
    \item profile based
    \item hybrid modes
      \begin{itemize}
        \item per object mode
        \item learning mode
      \end{itemize}
  \end{itemize}
\end{frame}

\begin{frame}{SMACK}
  \begin{itemize}
    \item inode based
    \item uses labels (most are kept in extended attribute -- \texttt{xattrs})
    \item policy based
    \item access
      \begin{itemize}
        \item rwxa - same as DAC
        \item t - transmutation
        \item b - report in bringup mode
      \end{itemize}
    \item custom labels: \_ \^ * ? @
  \end{itemize}
\end{frame}


\section{Sandboxing}

\begin{frame}{Assets to Protect}
  \begin{itemize}
    \item file descriptors
    \item file system space
    \item other processes
    \item memory
    \item network
    \item everything else
  \end{itemize}
\end{frame}

\begin{frame}{Sandbox Implementations}
  \begin{itemize}
    \item capabilities
    \item jail
    \item rule based (MAC)
    \item Java Virtual Machine
    \item HTML5 iframe sandbox
    \item .NET Code Access Security
  \end{itemize}
\end{frame}

\begin{frame}{Breaking Sandboxing}
  \begin{itemize}
    \pause \item faulty sandbox rules
    \pause \item other faulty configuration
    \pause \item kernel vulnerability
  \end{itemize}
\end{frame}

\begin{frame}{Linux Seccomp}
  \begin{itemize}
    \item minimize the exposed kernel surface
    \item to be used by developers
    \item uses BPF (\textit{Berkeley Packet Filtering})
    \item requires support in kernel
  \end{itemize}
\end{frame}

\begin{frame}{Kernel Config}
\begin{itemize}
  \item CONFIG_HAVE_ARCH_SECCOMP_FILTER=y
  \item CONFIG_SECCOMP_FILTER=y
  \item CONFIG_SECCOMP=y
\end{itemize}
\end{frame}

\begin{frame}{Default Allowed Syscalls}
  \begin{itemize}
      \item read
      \item write
      \item exit
      \item sigreturn
  \end{itemize}
\end{frame}

%\begin{frame}{Seccomp Example}
%  \begin{itemize}
%    \item povestea cu limbile
%  \end{itemize}
%\end{frame}


\begin{frame}{Android Application Sandbox}
  \begin{itemize}
    \pause \item \textit{The sandbox is simple, auditable, and based on decades-old UNIX-style user separation of processes and file permissions.}
    \pause \item SELinux-based
    \pause \item uses application UID to map sandbox to application
  \end{itemize}
\end{frame}

\section{Apple Application Sandbox}

\begin{frame}{Sandbox Profiles}
  \begin{itemize}
    \pause \item set of rules
    \pause \item sandbox operations, sandbox filters
    \pause \item provided as binary blobs in the kernel image
    \pause \item attached to an application
    \pause \item some apps may use the same sandbox profile
    \pause \item some system services use no sandbox profile
    \pause \item entitlement-checks and sandbox extensions for differentiation between apps using same sandbox profile
  \end{itemize}
\end{frame}

\begin{frame}{container Sandbox Profile}
  \begin{itemize}
    \pause \item default sandbox profiles for \textbf{all} 3rd party apps
    \pause \item biggest sandbox profile
  \end{itemize}
\end{frame}

\begin{frame}{SandScout}
  \begin{itemize}
    \pause \item \url{https://dl.acm.org/citation.cfm?id=2978336}
    \pause \item SandScout: Automatic Detection of Flaws in iOS Sandbox Profiles
    \pause \item systematic analysis of container sandbox profiles
    \pause \item found flaws: application collusion, device abuse, control bypass
  \end{itemize}
\end{frame}

\section{Summary}

\begin{frame}{Keywords}
  \begin{columns}
    \begin{column}{0.5\textwidth}
      \begin{itemize}
        \item access control
        \item Linux Security Module
        \item subject, object, permission
        \item capabilities
        \item profiles
      \end{itemize}
    \end{column}
    \begin{column}{0.5\textwidth}
      \begin{itemize}
        \item MAC
        \item SELinux, AppArmor, SMACK
        \item seccomp
        \item iOS sandboxing
        \item privacy settings
      \end{itemize}
    \end{column}
  \end{columns}
\end{frame}

%\begin{frame}{Resources}
%  \begin{itemize}
%    \item \href{https://link.url.com/}{Link Name}
%    \item TODO
%    \item TODO
%  \end{itemize}
%\end{frame}
%
%\begin{frame}{Nice to Read}
%  \begin{itemize}
%    \item TODO
%  \end{itemize}
%\end{frame}

\end{document}
