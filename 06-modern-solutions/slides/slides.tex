\documentclass{curs}

% Comment out lines below in case of no code to be included.
%\usepackage{code/highlight}
\usepackage{color}
\usepackage{graphicx}
%\usepackage{alltt}


\title[Session 06]{Session 06}
\subtitle{Modern Offensive and Defensive Solutions}
\author{Security of Information Systems (SIS)}
\date{November 2, 2018}

\begin{document}

\frame{\titlepage}

\begin{frame}{Attack and Defense}
  \begin{itemize}
    \pause \item attack: exploit vulnerabilities
    \pause \item defense: prevent attacks, make attacks difficult, confine attacks
    \pause \item attacker needs to find one security hole
    \pause \item defender has to protect all security holes
    \pause \item attacker invests time
    \pause \item defense mechanisms incur overhead
  \end{itemize}
\end{frame}

\begin{frame}{Attacker Perspective and Mindset}
  \begin{itemize}
    \pause \item find one vulnerability and build from that
    \pause \item look for something that is valuable
    \pause \item do reconnaisance, look for weak spots
    \pause \item create an attack chain
    \pause \item use every trick in the book
    \pause \item start from existing knowledge
  \end{itemize}
\end{frame}

\begin{frame}{Defender Perspective and Mindset}
  \begin{itemize}
    \pause \item protect all entry points
    \pause \item users are vulnerable, as well as technology
    \pause \item use multiple defensive layers
    \pause \item monitor, be proactive
    \pause \item discipline, best practices are worth more than skills
    \pause \item invest more on valuable targets
  \end{itemize}
\end{frame}

\begin{frame}{Attacker Pros/Cons}
  \begin{itemize}
    \pause \item apart from ethical hackers, security researchers, it's a shady business
    \pause \item you may not need skills, just a weak target and a database of attack vectors
    \pause \item you may get caught
    \pause \item you only need to find one spot
    \pause \item possible great gains
    \pause \item little time for fame (annonymous)
    \pause \item the Internet gives you tons of targets
    \pause \item but many targets give little more than fun
  \end{itemize}
\end{frame}

\begin{frame}{Defender Pros/Cons}
  \begin{itemize}
    \pause \item less resources (time) than an attacker
    \pause \item must think of everything
    \pause \item is being paid constructively
    \pause \item you have a purpose: keep the system running
    \pause \item it never ends
  \end{itemize}
\end{frame}

\begin{frame}{Honeypots}
  \begin{itemize}
    \pause \item baits
    \pause \item a system appearing as vulnerable but closely monitored
    \pause \item deflect, change attention and collect attacker information
  \end{itemize}
\end{frame}

\section{Modern Application Security}

\begin{frame}{Evolution of Application Security}
  \begin{itemize}
    \pause \item buffer overflows
    \pause \item shellcodes
    \pause \item memory protection (DEP, W\^X)
    \pause \item memory randomization
    \pause \item canaries
    \pause \item code reuse
    \pause \item CFI (Control Flow Integrity)
    \pause \item memory safety, safe programming languages
    \pause \item static and dynamic analysis
    \pause \item hardware enhanced security
  \end{itemize}
\end{frame}

\begin{frame}{Fine-grained ASLR}
  \begin{itemize}
    \pause \item \url{https://dl.acm.org/citation.cfm?id=2498135}
    \pause \item issue with ASLR: memory disclosure / information leak
    \pause \item one address leaked reveals all information
    \pause \item do it at page level
    \pause \item one leak may lead to other leaks that are chained together
  \end{itemize}
\end{frame}

\begin{frame}{SafeStack}
  \begin{itemize}
    \pause \item \url{https://clang.llvm.org/docs/SafeStack.html}
    \pause \item part of the Code Pointer Integrity project: \url{http://dslab.epfl.ch/proj/cpi/}
    \pause \item moves sensitive information (such as return addresses) on a safe stack, leaving problematic ones on the unsafe stack
    \pause \item reduced overhead, protects against stack buffer overflows
    \pause \item microStache: \url{https://www.springerprofessional.de/en/microstache-a-lightweight-execution-context-for-in-process-safe-/16103742}
  \end{itemize}
\end{frame}

\begin{frame}{Address Sanitizer}
  \begin{itemize}
    \pause \item ASan
    \pause \item \url{https://research.google.com/pubs/archive/37752.pdf}
    \pause \item \url{https://github.com/google/sanitizers/}
    \pause \item instruments code
    \pause \item only useful in development
    \pause \item detects out-of-bounds bugs, memory leaks
  \end{itemize}
\end{frame}

\begin{frame}{CFI/CPI}
  \begin{itemize}
    \pause \item \url{https://dl.acm.org/citation.cfm?id=1102165}
    \item \url{https://www.usenix.org/node/186160}
    \item \url{http://dslab.epfl.ch/proj/cpi/}
    \pause \item coarse-grained CFI vs fine-grained CFI
    \pause \item Control Flow Integrity, Code Pointer Integrity
    \pause \item protect against control flow hijack attacks
    \pause \item CPI is weaker than CFI but more practical (reduced overhead)
    \pause \item CPI protects all code pointers, data based attacks may still happen
    \pause \item CPS (Code Pointer Separation) is a weaker yet more practical for of CPI
  \end{itemize}
\end{frame}

\begin{frame}{Shellcodes}
  \begin{itemize}
    \pause \item difficult to inject due to DEP, small buffers and input validation
    \pause \item preliminary parts of the attack may remap memory region
    \pause \item shellcode may do stack pivoting and then load another shellcode
    \pause \item alphanumeric shellcodes: still need a binary address to bootstrap
  \end{itemize}
\end{frame}

\begin{frame}{Code Reuse}
  \begin{itemize}
    \pause \item bypass DEP by using existing pieces of code
    \pause \item code gadgets
    \pause \item used in ROP (\textit{Return-Oriented Programming}) and JOP (\textit{Jump-Oriented programming})
  \end{itemize}
\end{frame}

\begin{frame}{Return-Oriented Programming}
  \begin{itemize}
    \pause \item gadgets ending in \texttt{ret}
    \pause \item may be chained together to form an attack
    \pause \item Turing-complete languge
    \pause \item \url{http://www.suse.de/~krahmer/no-nx.pdf}
    \pause \item \url{https://dl.acm.org/citation.cfm?id=2133377}
    \pause \item most common way of creating runtime attack vectors
    \pause \item JOP: \url{https://dl.acm.org/citation.cfm?id=1966919}
      \begin{itemize}
        \pause \item gadgets end up in an indirect branch not a \texttt{ret}
      \end{itemize}
  \end{itemize}
\end{frame}

\begin{frame}{Anti-ROP Defense}
  \begin{itemize}
    \pause \item prevent atacks
      \begin{itemize}
        \pause \item SafeStack
        \pause \item CFI/CPI, ASan
        \pause \item Microsoft CFG, RFG
      \end{itemize}
    \pause \item detect attacks
      \begin{itemize}
        \pause \item Microsoft EMET (\textit{Enhanced Mitigation Experience Toolkit}), ProcessMitigations module
      \end{itemize}
  \end{itemize}
\end{frame}

\begin{frame}{Data-Oriented Attacks}
  \begin{itemize}
    \pause \item \url{https://www.usenix.org/node/190963}
    \pause \item \url{https://huhong-nus.github.io/advanced-DOP/}
    \pause \item overwrites data, not code pointers
    \pause \item bypasses CFI
  \end{itemize}
\end{frame}

\section{Modern OS Security}

\begin{frame}{Evolution of OS Security}
  \begin{itemize}
    \pause \item traditional main goals: functionality and reduced overhead
    \pause \item recent focus on OS security: plethora of devices and use cases
    \pause \item malware may easily take place among legitimate applications
    \pause \item kernel exploits become more common
    \pause \item OS virtualization, reduce TCB to hypervisor
    \pause \item include hardware-enforced security features
  \end{itemize}
\end{frame}

\begin{frame}{Mandatory Access Control}
  \begin{itemize}
    \pause \item opposed to Discretionary Access Control, where owner controls permissions
    \pause \item system-imposed settings
    \pause \item increased, centralized security
    \pause \item difficult to configure and maintain
    \pause \item rigid, non-elastic
    \pause \item Bell-LaPadula Model: \url{http://csrc.nist.gov/publications/history/bell76.pdf}
    \pause \item SELinux, TrustedBSD, Mandatory Integrity Control
  \end{itemize}
\end{frame}

\begin{frame}{Role-Based Access Control}
  \begin{itemize}
    \pause \item \url{https://ieeexplore.ieee.org/abstract/document/485845}
    \pause \item \url{https://dl.acm.org/citation.cfm?id=266751}
    \pause \item aggregate permissions into roles
    \pause \item role assignment, role authorization, permission authorization
    \pause \item useful in organizations
  \end{itemize}
\end{frame}

\begin{frame}{Sandboxing}
  \begin{itemize}
    \pause \item assume application may be malware
    \pause \item reduce potential damage
    \pause \item confine access to a minimal set of allowed actions
    \pause \item typically implemented at sandbox level (kernel enforced)
    \pause \item iOS sandboxing, Linux seccomp
  \end{itemize}
\end{frame}

\begin{frame}{Application Signing}
  \begin{itemize}
    \pause \item ensure application is validated
    \pause \item used by application stores and repositories: GooglePlay, Apple AppStore
    \pause \item device may not run non-signed apps
  \end{itemize}
\end{frame}

\begin{frame}{iOS Jekyll Apps}
  \begin{itemize}
    \pause \item \url{https://www.usenix.org/conference/usenixsecurity13/technical-sessions/presentation/wang_tielei}
    \pause \item apparently legimitate iOS app
    \pause \item bypasses Apple vetting
    \pause \item obfuscates calls to private libraries (part of the same address space, fixed from iOS 7)
    \pause \item once installed turns out to be malware
    \pause \item exfiltrates private data, exploits vulnerabilities
  \end{itemize}
\end{frame}

\begin{frame}{Jailreaking/Rooting}
  \begin{itemize}
    \pause \item \url{https://dl.acm.org/citation.cfm?id=3196527}
    \pause \item get root access on a device
    \pause \item close to full control
    \pause \item requires a critical vulnerability that gets root access
    \pause \item tethered (requires re-jailbreaking after reboot) cs non-tethered
    \pause \item essential for security researchers
  \end{itemize}
\end{frame}

\begin{frame}{Hardware-centric Attacks}
  \begin{itemize}
    \pause \item side channels
    \pause \item undocumented hardware features
    \pause \item imperfect hardware features that leak information
    \pause \item proprietary features that get exploited
    \pause \item hardware is part of TCB, reveals kernel memory
  \end{itemize}
\end{frame}

\begin{frame}{Sidechannel Attacks}
  \begin{itemize}
    \pause \item do not exploit vulnerabilities in applications or kernel code
    \pause \item mostly use features such as 
  \end{itemize}
\end{frame}

\begin{frame}{x86 Instruction Fuzzing}
  \begin{itemize}
    \pause \item \url{https://www.blackhat.com/docs/us-17/thursday/us-17-Domas-Breaking-The-x86-Instruction-Set-wp.pdf}
    \pause \item \url{https://github.com/xoreaxeaxeax/sandsifter}
    \pause \item instruction of length N is placed at the end of the page
    \pause \item creates a fuzzer for the x86 instruction set
    \pause \item found glitches, hidden instructions
  \end{itemize}
\end{frame}

\begin{frame}{IME}
  \begin{itemize}
    \pause \item \textit{Intel Management Engine}
    \pause \item AMD Secure Techonology
    \pause \item hardware features and highly proprietary firmware
    \pause \item \url{https://www.theverge.com/2018/1/3/16844630/intel-processor-security-flaw-bug-kernel-windows-linux}
    \pause \item user space app could access kernel space memory access
    \pause \item accused of being a backdoor to the system
  \end{itemize}
\end{frame}

\begin{frame}{rowhammer Attack}
  \begin{itemize}
    \pause \item \url{https://users.ece.cmu.edu/~yoonguk/papers/kim-isca14.pdf}
    \pause \item \url{https://www.vusec.net/projects/drammer/}
    \pause \item \url{https://googleprojectzero.blogspot.com/2015/03/exploiting-dram-rowhammer-bug-to-gain.html}
    \pause \item \url{https://www.blackhat.com/docs/us-15/materials/us-15-Seaborn-Exploiting-The-DRAM-Rowhammer-Bug-To-Gain-Kernel-Privileges.pdf}
    \pause \item hardware fault in DRAM chips
    \pause \item constant bit flip pattern in certain rows can cause a flip in another row (not belonging to the current process)
    \pause \item may be exploited to get root access
  \end{itemize}
\end{frame}

\begin{frame}{Spectre and Meltdown}
  \begin{itemize}
    \pause \item \url{https://meltdownattack.com}
    \pause \item \url{https://www.usenix.org/system/files/conference/usenixsecurity18/sec18-lipp.pdf}
    \pause \item application may access data from another application
    \pause \item Meltdown exploits a hardware race condition allowing an unprivileged process to read privileged data
    \pause \item Spectre does a side channel attack on speculative execution features of modern CPUs
    \pause \item hardware fixes by Intel, software solutions
  \end{itemize}
\end{frame}

\begin{frame}{KPTI}
  \begin{itemize}
    \pause \item \textit{Kernel Page Table Isolation}
    \pause \item \url{https://lwn.net/Articles/741878/}
    \pause \item place kernel in separate address space
    \pause \item mitigation against hardware-centric attacks
  \end{itemize}
\end{frame}

\begin{frame}{Hypervisor Attacks}
  \begin{itemize}
    \pause \item \url{https://dl.acm.org/citation.cfm?id=2484406}
    \pause \item attack/compromise hypervisor, get control of VMs
    \pause \item may exploit a vulnerability in the hypercall interface or may exploit a hardware bug
    \pause \item hyperjacking
  \end{itemize}
\end{frame}

\section{Modern Web Security}

\begin{frame}{Evolution of Web Security}
  \begin{itemize}
    \pause \item path traversals, misconfigurations
    \pause \item injections
    \pause \item XSS
    \pause \item misconfiguration
    \pause \item unsafe communication
    \pause \item application/language bugs
  \end{itemize}
\end{frame}

\begin{frame}{Secure Communication}
  \begin{itemize}
    \pause \item provide secure communication between client and server
    \pause \item HTTPS everywhere
    \pause \item Secure Cookie
    \pause \item strong encryption, strong protocols
  \end{itemize}
\end{frame}

\begin{frame}{Attacks on Security Protocols}
  \begin{itemize}
    \pause \item \url{https://tools.ietf.org/html/rfc7457}
    \pause \item \url{https://www.mitls.org/pages/attacks}
    \pause \item flaws in protocol logic
    \pause \item cryptographic design flaws
    \pause \item implementation flaws
  \end{itemize}
\end{frame}

\begin{frame}{Connection Downgrade}
  \begin{itemize}
    \pause \item part of man-in-the-middle attack
    \pause \item negociate a connection with weaker protocol features than the current one
    \pause \item ideally drop HTTPS alltogether
    \pause \item POODLE (\textit{Padding Oracle On Downgraded Legacy Encryption})
    \pause \item \url{https://www.openssl.org/~bodo/ssl-poodle.pdf}
  \end{itemize}
\end{frame}

\begin{frame}{Advanced Injection Attacks}
  \begin{itemize}
    \pause \item LDAP, XPath injection
    \pause \item blind SQL injection: content-based and time-based
    \pause \item \url{https://www.owasp.org/images/7/74/Advanced_SQL_Injection.ppt}
    \pause \item \url{https://nvisium.com/blog/2015/06/17/advanced-sql-injection.html}
  \end{itemize}
\end{frame}

\begin{frame}{Language Bugs}
  \begin{itemize}
    \pause \item bugs/vulnerabilities in frameworks
    \pause \item bugs/vulnerabilities in web modules or languate interpreter
  \end{itemize}
\end{frame}

\section{Summary}

\begin{frame}{Modern Offensive and Defensive Techniques}
  \begin{itemize}
    \pause \item attacks focus on low-level aspects of a system: hide features, exploit hardware, side channels, protocol design
    \pause \item assume better/improved applications but imperfect system/protocol/configuration design
    \pause \item defense takes more time and incurs significant overhead
    \pause \item battle rages on
  \end{itemize}
\end{frame}

\begin{frame}{Keywords}
  \begin{columns}
    \begin{column}{0.5\textwidth}
      \begin{itemize}
        \item honeypot
        \item fine-grained ASLR
        \item SafeStack
        \item AddressSanitizer
        \item CFI/CPI
        \item code reuse
        \item ROP, JOP
        \item data-oriented attacks
        \item MAC, RBAC
        \item sandboxing
      \end{itemize}
    \end{column}
    \begin{column}{0.5\textwidth}
      \begin{itemize}
        \item Jekyll apps
        \item jailbreak, rooting
        \item side channel attacks
        \item IME attack
        \item Meltdown, Spectre
        \item KPTI
        \item rowhammer
        \item connection downgrade
        \item POODLE
        \item blind SQL injection
      \end{itemize}
    \end{column}
  \end{columns}
\end{frame}

\begin{frame}{Resources}
  \begin{itemize}
    \item see URLs accross slides
  \end{itemize}
\end{frame}

\end{document}
