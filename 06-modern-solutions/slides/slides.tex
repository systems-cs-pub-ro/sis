\documentclass{curs}

% Comment out lines below in case of no code to be included.
%\usepackage{code/highlight}
\usepackage{color}
\usepackage{graphicx}
%\usepackage{alltt}


\title[Session 06]{Session 06}
\subtitle{Modern Offensive and Defensive Solutions}
\author{Security of Information Systems (SIS)}
\date{November 10, 2017}

\begin{document}

\frame{\titlepage}

\begin{frame}{Attack and Defense}
  \begin{itemize}
    \item attack: exploit vulnerabilities
    \item defense: prevent attacks, make attacks difficult, confine attacks
    \item attacker needs to find one security hole
    \item defender has to protect all security holes
    \item attacker invests time
    \item defense mechanisms incur overhead
  \end{itemize}
\end{frame}

\begin{frame}{Attacker Perspective and Mindset}
  \begin{itemize}
    \item find one vulnerability and build from that
    \item look for something that is valuable
    \item do reconnaisance, look for weak spots
    \item create an attack chain
    \item use every trick in the book
    \item start from existing knowledge
  \end{itemize}
\end{frame}

\begin{frame}{Defender Perspective and Mindset}
  \begin{itemize}
    \item protect all entry points
    \item users are vulnerable, as well as technology
    \item use multiple defensive layers
    \item monitor, be proactive
    \item discipline, best practices are worth more than skills
    \item invest more on valuable targets
  \end{itemize}
\end{frame}

\begin{frame}{Attacker Pros/Cons}
  \begin{itemize}
    \item apart from ethical hackers, security researchers, it's a shady business
    \item you may not need skills, just a weak target and a database of attack vectors
    \item you may get caught
    \item you only need to find one spot
    \item possible great gains
    \item little time for fame (annonymous)
    \item the Internet gives you tons of targets
    \item but many targets give little more than fun
  \end{itemize}
\end{frame}

\begin{frame}{Defender Pros/Cons}
  \begin{itemize}
    \item less resources (time) than an attacker
    \item must think of everything
    \item is being paid constructively
    \item you have a purpose: keep the system running
    \item it never ends
  \end{itemize}
\end{frame}

\begin{frame}{Honeypots}
  \begin{itemize}
    \item baits
    \item a system appearing as vulnerable but closely monitored
    \item deflect, change attention and collect attacker information
  \end{itemize}
\end{frame}

\section{Modern Application Security}

\begin{frame}{Evolution of Application Security}
  \begin{itemize}
    \item buffer overflows
    \item shellcodes
    \item memory protection (DEP, W\^X)
    \item memory randomization
    \item canaries
    \item code reuse
    \item CFI (Control Flow Integrity)
    \item memory safety, safe programming languages
    \item static and dynamic analysis
    \item hardware enhanced security
  \end{itemize}
\end{frame}

\begin{frame}{Fine-grained ASLR}
  \begin{itemize}
    \item issue with ASLR: memory disclosure / information leak
    \item one address leaked renders all information
    \item do it at page level
    \item one leak may lead to other leaks that are chained together
    \item \url{https://dl.acm.org/citation.cfm?id=2498135}
  \end{itemize}
\end{frame}

\begin{frame}{SafeStack}
  \begin{itemize}
    \item \url{https://clang.llvm.org/docs/SafeStack.html}
    \item part of the Code Pointer Integrity project: \url{http://dslab.epfl.ch/proj/cpi/}
    \item moves sensitive information (such as return addresses) on a safe stack, leaving problematic ones on the unsafe stack
    \item reduced overhead, protects against stack buffer overflows
  \end{itemize}
\end{frame}

\begin{frame}{Address Sanitizer}
  \begin{itemize}
    \item ASan
    \item \url{https://research.google.com/pubs/archive/37752.pdf}
    \item \url{https://github.com/google/sanitizers/}
    \item instruments code
    \item only useful in development
    \item detects out-of-bounds bugs, memory leaks
  \end{itemize}
\end{frame}

\begin{frame}{CFI/CPI}
  \begin{itemize}
    \item \url{https://dl.acm.org/citation.cfm?id=1102165}
    \item \url{https://www.usenix.org/node/186160}
    \item \url{http://dslab.epfl.ch/proj/cpi/}
    \item coarse-grained CFI vs fine-grained CFI
    \item Control Flow Integrity, Code Pointer Integrity
    \item protect against control flow hijack attacks
    \item CPI is weaker than CFI but more practical (reduced overhead)
    \item CPI protects all code pointers, data based attacks may still happen
    \item CPS (Code Pointer Separation) is a weaker yet more practical for of CPI
  \end{itemize}
\end{frame}

\begin{frame}{Shellcodes}
  \begin{itemize}
    \item difficult to inject due to DEP, small buffers and input validation
    \item preliminary parts of the attack may remap memory region
    \item shellcode may do stack pivoting and then load another shellcode
    \item alphanumeric shellcodes: still need a binary address to bootstrap
  \end{itemize}
\end{frame}

\begin{frame}{Code Reuse}
  \begin{itemize}
    \item bypass DEP by using existing pieces of code
    \item code gadgets
    \item used in ROP (Return-Oriented Programming) and JOP (Jump-Oriented programming)
  \end{itemize}
\end{frame}

\begin{frame}{Return-Oriented Programming}
  \begin{itemize}
    \item gadgets ending in \texttt{ret}
    \item may be chained together to form an attack
    \item Turing-complete languge
    \item \url{http://www.suse.de/~krahmer/no-nx.pdf}
    \item \url{https://dl.acm.org/citation.cfm?id=2133377}
    \item most common way of creating runtime attack vectors
    \item JOP: \url{https://dl.acm.org/citation.cfm?id=1966919}
      \begin{itemize}
        \item gadgets end up in an indirect branch not a \texttt{ret}
      \end{itemize}
  \end{itemize}
\end{frame}

\begin{frame}{Anti-ROP Defense}
  \begin{itemize}
    \item prevent atacks
      \begin{itemize}
        \item SafeStack
        \item CFI/CPI, ASan
        \item Microsoft CFG, RFG
      \end{itemize}
    \item detect attacks
      \begin{itemize}
        \item Microsoft EMET
      \end{itemize}
  \end{itemize}
\end{frame}

\begin{frame}{Data-Oriented Attacks}
  \begin{itemize}
    \item \url{https://www.usenix.org/node/190963}
    \item \url{https://huhong-nus.github.io/advanced-DOP/}
    \item overwrites data, not code pointers
    \item bypasses CFI
  \end{itemize}
\end{frame}

\section{Modern OS Security}

\begin{frame}{Evolution of OS Security}
  \begin{itemize}
    \item TODO
  \end{itemize}
\end{frame}

\begin{frame}{Mandatory Access Control}
  \begin{itemize}
    \item TODO
  \end{itemize}
\end{frame}

\begin{frame}{Sandboxing}
  \begin{itemize}
    \item TODO
  \end{itemize}
\end{frame}

\begin{frame}{Application Signing}
  \begin{itemize}
    \item TODO
  \end{itemize}
\end{frame}

\begin{frame}{iOS Jekyll Apps}
  \begin{itemize}
    \item TODO
  \end{itemize}
\end{frame}

\begin{frame}{Jailreaking/Rooting}
  \begin{itemize}
    \item TODO
  \end{itemize}
\end{frame}

\begin{frame}{Hardware-centric Attacks}
  \begin{itemize}
    \item TODO
  \end{itemize}
\end{frame}

\begin{frame}{x86 Instruction Fuzzing}
  \begin{itemize}
    \item TODO
  \end{itemize}
\end{frame}

\begin{frame}{IME}
  \begin{itemize}
    \item TODO
  \end{itemize}
\end{frame}

\begin{frame}{Sidechannel Attacks}
  \begin{itemize}
    \item TODO
  \end{itemize}
\end{frame}

\begin{frame}{rowhammer Attack}
  \begin{itemize}
    \item TODO
  \end{itemize}
\end{frame}

\begin{frame}{Hypervisor Attacks}
  \begin{itemize}
    \item TODO
  \end{itemize}
\end{frame}

\section{Modern Web Security}

\begin{frame}{Evolution of Web Security}
  \begin{itemize}
    \item TODO
  \end{itemize}
\end{frame}

\begin{frame}{Secure Communication}
  \begin{itemize}
    \item TODO
  \end{itemize}
\end{frame}

\begin{frame}{Attacks on Security Protocols}
  \begin{itemize}
    \item TODO
  \end{itemize}
\end{frame}

\begin{frame}{Connection Downgrade}
  \begin{itemize}
    \item TODO
  \end{itemize}
\end{frame}

\begin{frame}{Advanced Injection Attacks}
  \begin{itemize}
    \item TODO
  \end{itemize}
\end{frame}

\begin{frame}{Language Bugs}
  \begin{itemize}
    \item TODO
  \end{itemize}
\end{frame}

\section{Summary}

\begin{frame}{TODO}
  \begin{itemize}
    \item TODO
  \end{itemize}
\end{frame}

\begin{frame}{Keywords}
  \begin{columns}
    \begin{column}{0.5\textwidth}
      \begin{itemize}
        \item TODO
      \end{itemize}
    \end{column}
    \begin{column}{0.5\textwidth}
      \begin{itemize}
        \item TODO
      \end{itemize}
    \end{column}
  \end{columns}
\end{frame}

\begin{frame}{Resources}
  \begin{itemize}
    \item \href{https://link.url.com/}{Link Name}
    \item TODO
    \item TODO
  \end{itemize}
\end{frame}

\begin{frame}{Nice to Read}
  \begin{itemize}
    \item TODO
  \end{itemize}
\end{frame}

\end{document}
