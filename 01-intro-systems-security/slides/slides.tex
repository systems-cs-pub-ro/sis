\documentclass{curs}

% Comment out lines below in case of no code to be included.
%\usepackage{code/highlight}
%\usepackage{color}
%\usepackage{alltt}


\title[Session 01]{Session 01}
\subtitle{Introduction to Systems Security}
\author{Security of Information Systems (SIS)}
\date{September 28, 2018}

\begin{document}

\frame{\titlepage}


\section{Systems Security}

\begin{frame}{Systems Security}
  \begin{itemize}
    \pause \item IT security, cyber security, computer security
    \pause \item applications, operating system, infrastructure, networking
    \pause \item hardware, software, information
  \end{itemize}
\end{frame}

\begin{frame}{Security of Information Systems}
  \begin{itemize}
    \pause \item focus on computer, operating system, software stack and infrastructure security
    \pause \item little focus on networking, crypto, hardware
    \pause \item research focus
  \end{itemize}
\end{frame}

\begin{frame}{Topics}
  \begin{itemize}
    \pause \item authentication, authorization
    \pause \item attack, mitigation, bypass
    \pause \item isolation, sandboxing
    \pause \item fuzzing, symbolic execution
    \pause \item defensive programming
    \pause \item software analysis, verification and validation
  \end{itemize}
\end{frame}

\begin{frame}{Systems Security Research}
  \begin{itemize}
    \pause \item research: more on finding the right question, than the right answer
    \pause \item not boring, but uncertain
    \pause \item high risk, high impact
    \pause \item secure build, attack, defend, repeat
    \pause \item actual systems: software, hardware, infrastructure
  \end{itemize}
\end{frame}

\begin{frame}{Systems Security Research Venues}
  \begin{itemize}
    \pause \item big four: IEEE S\&P (Oakland), ACM CCS, NDSS, Usenix Security Symposium
    \pause \item ESORICS, RAID, ACSAC
    \pause \item AsiaCCS, CNS, ACNS
    \pause \item \url{http://faculty.cs.tamu.edu/guofei/sec_conf_stat.htm}
    \pause \item \url{http://s3.eurecom.fr/~balzarot/notes/top4_v1/}
  \end{itemize}
\end{frame}

\begin{frame}{Research Results}
  \begin{itemize}
    \pause \item white papers: company product information
    \pause \item workshops: preliminary work, networking
    \pause \item conferences: complete work, novel results, networking
    \pause \item journals: aggregated work, archive
    \pause \item CVEs
    \pause \item features in existing products
    \pause \item start ups
    \pause \item public / open source
  \end{itemize}
\end{frame}

\section{Organization}

\begin{frame}{Logistics}
  \begin{itemize}
    \item lecture: Friday, 10am-12pm, room PR706
    \item labs: Friday, 8am-10am and 12pm-2pm, room PR706
    \item first lab is Friday, September 28, 2018, 12pm-2pm and Friday, October 5, 8am-10am
    \item \url{http://elf.cs.pub.ro/sis/}: all information
    \item \url{https://sis-ctf.security.cs.pub.ro}: CTF assignments
    \item acs.curs.pub.ro instance: discussions, annoucements
  \end{itemize}
\end{frame}

\begin{frame}{Team}
  \begin{itemize}
    \item Costin Carabaș
    \item Radu Chișcariu
    \item Alexandru Cornea
    \item Răzvan Deaconescu
  \end{itemize}
\end{frame}

\begin{frame}{Typical Lecture Session}
  \begin{itemize}
    \pause \item lecture + discussions
    \pause \item research insights, trends
    \pause \item demos, highlights
    \pause \item DO ask questions, DO take part in discussion, DO challenge ideas, DO rebut ideas, DO open interesting topics (covering systems security)
  \end{itemize}
\end{frame}

\begin{frame}{Typical Lab Session}
  \begin{itemize}
    \pause \item lab/practical focused on problem solving, on approaches, not on tools
    \pause \item CTF-like activity in lab: have problem, find solution
    \pause \item challenge focused, not tutorial focused
    \pause \item need focus, patience and perseverence
    \pause \item we expect you to know the basics of: Linux/command line, network protocols and infrastructure, programming, assembly language, operating systems
  \end{itemize}
\end{frame}

\begin{frame}{CNS vs. SIS}
  \begin{itemize}
    \pause \item focus on offensive/exploiting vs focused on a building, attacking, defending
    \pause \item narrow, focused, practical vs. diverse, holistic, conceptual
    \pause \item binary/runtime application security vs. systems security
    \pause \item engineer-centric vs. research-centric
    \pause \item well-known steps vs. trends and models
    \pause \item skill vs. approach
  \end{itemize}
\end{frame}

\begin{frame}{Support Documentation}
  \begin{itemize}
    \item still looking for proper documentation
    \item it's likely it will not be a book
    \item likely 3-4 papers and/or chapters per session
  \end{itemize}
\end{frame}

\begin{frame}{Grading}
  \begin{itemize}
    \item 2p: lecture tests, once every two lectures (3, 5, 7, 9, 11)
    \item 2p: lab activity
    \item 1p: lecture activity (DO take part in lectures)
    \item 3p: assignments/projects (bonuses included)
    \item 3p: final exam (2 essay-like questions out of 3 presented)
  \end{itemize}
\end{frame}

\begin{frame}{Contents}
  \begin{enumerate}
    \pause \item Introduction to Systems Security
    \pause \item Authentication and Password Management
    \pause \item Attacks and Exploits
    \pause \item Mitigation Techniques
    \pause \item Defense Bypassing
    \pause \item Application Confinment
    \pause \item System Isolation
    \pause \item Information Flow Security
    \pause \item Static Analysis. Defensive Programming
    \pause \item Fuzzing. Symbolic Execution
    \pause \item Software Verification and Validation
  \end{enumerate}
\end{frame}

\begin{frame}{Storyline}
  \begin{itemize}
    \pause \item we have a system/infrastucture
    \pause \item one way to attack it is to crack or abuse authentication
    \pause \item assuming valid entry points (i.e. authentication) are secure, the attacker looks for security holes (i.e. vulnerabilities)
    \pause \item the attacker exploits vulnerabilities, we use defensive mechanisms, the attacker tries to bypass them
    \pause \item we assume attacks are unavoidable, we aim to confine them; we confine applications
    \pause \item since applications interact with other applications and system components, we isolate the system
    \pause \item proper isolation relies on separation and marking roles and validating the information flow
  \end{itemize}
\end{frame}

\begin{frame}{Storyline (cont.)}
  \begin{itemize}
    \pause \item ideally, our application or system will be secure by specification, design, implementation, validated offline
    \pause \item we use static analysis and dynamic analysis (including fuzzing and symbolic execution) to validate an application offline
    \pause \item a thorough approach is to use (semi)formal validation on application specifications and build it securely from those
  \end{itemize}
\end{frame}

\begin{frame}{Prerequisites}
  \begin{itemize}
    \pause \item good understanding of operating systems
    \pause \item good Unix/Linux command line abilities
    \pause \item good C programming skills
    \pause \item fair knowledge of C shell scripting
    \pause \item fair knowledge of Unix/Linux development tools
    \pause \item basic understanding networking
    \pause \item basic knowledge of computer architecture and assembly language
  \end{itemize}
\end{frame}

\begin{frame}{Covering Missing Know-How}
  \begin{itemize}
    \item Linux/command line: USO labs: \url{http://ocw.cs.pub.ro/courses/uso}
    \item networking: RL labs: \url{http://ocw.cs.pub.ro/courses/rl}
    \item assembly: \url{http://ocw.cs.pub.ro/courses/iocla}
    \item operating systems: \url{http://ocw.cs.pub.ro/courses/so}
  \end{itemize}
\end{frame}


\section{Systems Security Concepts}

\begin{frame}{Scope}
  \begin{itemize}
    \item application
    \item operating system
    \item hardware
    \item I/O
    \item networking, infrastructure
  \end{itemize}
\end{frame}

\begin{frame}{Security Objectives}
  \begin{itemize}
    \item privacy
    \item safety
    \item anonimity
    \item integrity
    \item confidentiality
    \item availability
  \end{itemize}
\end{frame}

\begin{frame}{Security vs. Safety}
  \begin{itemize}
    \item safety is state
    \item security is process
    \item safety is outcome of a secure process/procedure
    \item security: umbrella, safety: warm and not getting wet
  \end{itemize}
\end{frame}

\begin{frame}{Trust}
  \begin{itemize}
    \item validation among parties
    \item chain of trust: bottom-up trust
    \item trust anchor
    \item certificate chain, digital signature, certification authority
  \end{itemize}
\end{frame}

\begin{frame}{A*}
  \begin{itemize}
    \item authentication: letting an entity be part of the system
    \item authorization: providing privileges to a given entity (subject) for resources (object)
    \item access control: checking whether subject can access object
    \item audit: inspecting action history, validating behavior
  \end{itemize}
\end{frame}

\begin{frame}{A* commands}
  \begin{itemize}
    \item authentication: \texttt{login}
    \item authorization: \texttt{chmod}
    \item access control: \texttt{ls}
    \item audit: \texttt{find}, \texttt{auditd}
  \end{itemize}
\end{frame}

\begin{frame}{Least Privilege}
  \begin{itemize}
    \item each subject is only provided the permissions it needs
    \item if it's not mandatory, ditch it
    \item sandboxing, isolation
  \end{itemize}
\end{frame}

\begin{frame}{Privilege Separation/Escalation}
  \begin{itemize}
    \item split privileges among entities
    \item reduced actions for super user
    \item may require escalation (i.e. one subject may get priviliges from another)
      \begin{itemize}
        \item be really careful about that
        \item sudo, setuid, capabilities
      \end{itemize}
  \end{itemize}
\end{frame}

\begin{frame}{Trusted Computing Base}
  \begin{itemize}
    \item security-critical parts of the computer system
    \item bugs inside TCP compromise security
    \item i.e. priviliged parts of the computer system
    \item aim to reduce TCB
  \end{itemize}
\end{frame}


\section{Attack}

\begin{frame}{Attacker Mindset}
  \begin{itemize}
    \item maximize profit
    \item time is on your side
    \item brute force may be an option
    \item you don't care as long as it works
    \item target many, get one
  \end{itemize}
\end{frame}

\begin{frame}{Attacker Objectives}
  \begin{itemize}
    \item control
    \item cripple
    \item steal (information leak)
  \end{itemize}
\end{frame}

\begin{frame}{Threat/Adversarial Model}
  \begin{itemize}
    \item decompose application, identify threats, determine coutermeasures
    \item classify possible attacker actions and attacker flow
  \end{itemize}
\end{frame}

\begin{frame}{Vulnerability}
  \begin{itemize}
    \item misconfiguration, weakness
    \item exposes a risk that may be exploited for unintended outcome
    \item i.e. buffer overflow, integer overflow, unsanitized input
  \end{itemize}
\end{frame}

\begin{frame}{Exploit}
  \begin{itemize}
    \item method to use a vulnerability for malicious outcome
    \item take advantage
    \item i.e. ret-to-libc, ROP, send injection code
  \end{itemize}
\end{frame}

\begin{frame}{Attack Vector}
  \begin{itemize}
    \item steps/path to deliver a malicious outcome
    \item composed of attack steps (attack gadgets)
    \item usually chaining together exploits towards outcome
    \item i.e. SQL injection, actual real world attacks
  \end{itemize}
\end{frame}

\begin{frame}{Attack Surface}
  \begin{itemize}
    \item parts of the system exposed to attacks
    \item entry points into system
    \item either valid entry points that may be used invalidly
    \item or invalid entry points
    \item goal is to reduce attack surface (think TCB)
  \end{itemize}
\end{frame}


\section{Defense}

\begin{frame}{Defender Mindset}
  \begin{itemize}
    \item malicious vs. negligence
    \item time is limited
    \item proactive before reactive
    \item prevent
    \item monitor
    \item defense in depth
  \end{itemize}
\end{frame}

\begin{frame}{Proactive vs. Reactive}
  \begin{itemize}
    \item similar to medicine, treating disease
    \item analyze threats beforehand
    \item deploy mitigation solutions
    \item react when sh*t happens
    \item have solutions in place
  \end{itemize}
\end{frame}

\begin{frame}{Defense in Depth}
  \begin{itemize}
    \item multiple defense layers
    \item redundancy in defense
    \item assume one layer falls, others take its place
  \end{itemize}
\end{frame}

\begin{frame}{Phases for Defensive Mechanisms}
  \begin{itemize}
    \item harden
    \item prevent
    \item confine
    \item treat
  \end{itemize}
\end{frame}

\begin{frame}{Harden}
  \begin{itemize}
    \item analysis, enhancement
    \item off-line
    \item static, dynamic analysis
    \item verification and validation
    \item fuzzing, symbolic execution
    \item remove, treat \textbf{vulnerabilities}
  \end{itemize}
\end{frame}

\begin{frame}{Prevent}
  \begin{itemize}
    \item make it difficult / impossibile for exploits to happen
    \item on-line, during runtime
    \item i.e. ASLR, DEP, stack guard, ASan
  \end{itemize}
\end{frame}

\begin{frame}{Confine}
  \begin{itemize}
    \item when sh*t happens, reduce damage
    \item limit resources use, limit interface
    \item as reduced privileges as possible
    \item ideally customed (tailored) per app/system
    \item sandboxing, virtualization, access control
  \end{itemize}
\end{frame}

\begin{frame}{Treat}
  \begin{itemize}
    \item if sh*t happens, solve it
    \item remove threat, remove damaged resources
    \item replace components
    \item apply lessons learned
  \end{itemize}
\end{frame}

\begin{frame}{Monitor}
  \begin{itemize}
    \item know when sh*t happens ASAP
    \item proper monitoring levels; otherwise you ignore it
    \item monitor as much as you can
    \item key for availability
  \end{itemize}
\end{frame}


\section{Summary}

\begin{frame}{Keywords}
  \begin{columns}
    \begin{column}{0.5\textwidth}
      \begin{itemize}
        \item systems security
        \item Security of Information Systems
        \item goals
        \item safety
        \item trust
        \item trusted computing base
        \item least privilege
        \item threat model
      \end{itemize}
    \end{column}
    \begin{column}{0.5\textwidth}
      \begin{itemize}
        \item vulnerability
        \item exploit
        \item attack vector
        \item attack surface
        \item defense in depth
        \item harden
        \item prevent
        \item confine
        \item treat
        \item monitor
      \end{itemize}
    \end{column}
  \end{columns}
\end{frame}

\end{document}
