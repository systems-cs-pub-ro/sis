\documentclass{curs}

% Comment out lines below in case of no code to be included.
%\usepackage{code/highlight}
\usepackage{color}
\usepackage{graphicx}
%\usepackage{alltt}


\title[Session 03]{Session 03}
\subtitle{Exploiting. Part 1: Applications}
\author{Security of Information Systems (SIS)}
\date{October 11, 2019}

\begin{document}

\frame{\titlepage}

\begin{frame}{Attacking a System}
  \begin{enumerate}
    \pause \item steal (information leak, information disclosure)
    \pause \item control (access, privileges)
    \pause \item cripple (crash, denial of service, sabotage)
  \end{enumerate}
\end{frame}

\begin{frame}{Paths to Controlling a System}
  \begin{itemize}
    \pause \item breaking authentication
    \pause \item side channel attacks
    \pause \item bypass checks (misconfigurations)
    \pause \item exploit vulnerabilities
  \end{itemize}
\end{frame}

\begin{frame}{Breaking Authentication}
  \begin{itemize}
    \pause \item guess passwords (or other credentials)
    \pause \item crack passwords (or other credentials)
    \pause \item social engineering
    \pause \item impersonate
  \end{itemize}
\end{frame}

\begin{frame}{Side Channel Attacks}
  \begin{itemize}
    \pause \item do not alter or attack system directly
    \pause \item covert channel
    \pause \item infer information (passwords, keys, messages) from error messages, power dissipation, electromagnetic signals, sounds etc.
    \pause \item system-centric attack not application-centric attack: you may have a perfect app but a flawed system
  \end{itemize}
\end{frame}

\begin{frame}{Misconfigurations}
  \begin{itemize}
    \pause \item mostly wrong ACL checks
    \pause \item restricted information is available
    \pause \item caused by system complexity and/or programmer/designer/administrator lack of complete view of the system
  \end{itemize}
\end{frame}

\begin{frame}{Exploiting}
  \begin{itemize}
    \pause \item system/application has a vulnerability: can be used for attacker benefit
    \pause \item unintended behavior (not known or not checked by designer)
    \pause \item can get inside the system/application, can control the system/application
    \pause \item issue created by the designer/developer of the application/system
  \end{itemize}
\end{frame}

\section{Attacking a System. Malware}

\begin{frame}{Attacking a System}
  \begin{itemize}
    \pause \item find a ``way in'': misconfiguration, exploit
    \pause \item get as much power as possible (look for privilege escalation, go for complete privileges)
    \pause \item extract information
    \pause \item control the system
    \pause \item hide presence
    \pause \item make it persistent
  \end{itemize}
\end{frame}

\begin{frame}{Attack Vector}
  \begin{itemize}
    \pause \item steps for an attack
    \pause \item do reconnaissance, do information leak, get access, escalate, make permanent
    \pause \item different vulnerabilities or flaws are exploited in an exploit chain
  \end{itemize}
\end{frame}

\begin{frame}{Malware}
  \begin{itemize}
    \pause \item software with malicious intent
    \pause \item it's implanted on the target system, it runs on the target system
    \pause \item an exploit may be exploited remote or locally by malware
    \pause \item a separate attack must be used to implant the malware
  \end{itemize}
\end{frame}

\begin{frame}{Types of Malware}
  \begin{itemize}
    \pause \item \url{http://www.malwaretruth.com/the-list-of-malware-types/}
    \pause \item adware
    \pause \item spyware
    \pause \item virus
    \pause \item worm
    \pause \item trojan
    \pause \item rootkit
    \pause \item backdoor
    \pause \item keylogger
    \pause \item ransomware
  \end{itemize}
\end{frame}

\begin{frame}{System/Component Flows}
  \begin{itemize}
    \pause \item input $\rightarrow$ attack surface
    \pause \item input processing by applications $\rightarrow$ input validation
    \pause \item application uses internal control flow to process data
    \pause \item flaws/vulnerabilities may appear inside the app, or in the component interaction (access control lists, configuration files, message passing)
    \pause \item control flow vs. data flow
  \end{itemize}
\end{frame}

\section{Application Exploiting}

\begin{frame}{Application Exploiting}
  \begin{itemize}
    \pause \item vulnerability in app allows leak or control of app
    \pause \item generally related to memory exploiting: memory disclosure, memory overwrite
    \pause \item goals
      \begin{itemize}
        \pause \item critical data (read or overwrite)
        \pause \item code pointers (overwrite and alter control flow)
      \end{itemize}
  \end{itemize}
\end{frame}

\begin{frame}{Buffer Overflow}
  \begin{itemize}
    \pause \item most basic vulnerability
    \pause \item go past the buffer boundary and overwrite data
    \pause \item look for code pointers: return address on stack, function pointers
  \end{itemize}
\end{frame}

\section{Runtime Binary Application Attacks}

\begin{frame}{Runtime Binary Application Attacks}
  \begin{itemize}
    \pause \item exploit running application
    \pause \item identify vulnerability and corrupt memory
    \pause \item generally aim to control the app, run arbitrary code, get shell
    \pause \item ideal step is to get privileged access to the system
  \end{itemize}
\end{frame}

\begin{frame}{Attack Steps}
  \begin{itemize}
    \pause \item identify vulnerability (usually buffer overflow)
    \pause \item determine offset from the start of the buffer to target to overwrite (usually a code pointer)
    \pause \item determine value used to overwrite target (points to ``useful'' attacker code)
    \pause \item craft payload
      \begin{enumerate}
        \pause \item initial padding (size offset)
        \pause \item overwrite value
        \pause \item possible other values (function arguments, code gadgets)
        \pause \item possible initial shellcode
      \end{enumerate}
    \pause \item inject payload in vulnerable application
    \pause \item profit
  \end{itemize}
\end{frame}

\begin{frame}{Control-Flow Hijacking}
  \begin{itemize}
    \pause \item goal is to alter the control flow and take control of the program
    \pause \item we can create new edges in the control flow graph: code reuse (ROP)
    \pause \item we can add new vertices in the control flow graph: code injection (shellcode)
    \pause \item CFI (\textit{Control Flow Integrity}) is used to prevent control-flow highjacking: expensive
  \end{itemize}
\end{frame}

\begin{frame}{Data-Oriented Attacks}
  \begin{itemize}
    \pause \item overwrite data (no code pointers) and do not alter the control flow
    \pause \item use existing/valid paths in the control flow to get control of the program or leak information
    \pause \item Hong Hu, Shweta Shinde, Sendroiu Adrian, Zheng Leong Chua, Prateek Saxena, Zhenkai Liang: Data-Oriented Programming: On the Expressiveness of Non-Control Data Attacks, IEEE S\&P 2016
  \end{itemize}
\end{frame}

\section{Summary}

\begin{frame}{Keywords}
  \begin{columns}
    \begin{column}{0.5\textwidth}
      \begin{itemize}
        \item system components
        \item exploit
        \item vulnerability
        \item malware
        \item attack vector
        \item attack surface
        \item input validation
        \item code pointer
      \end{itemize}
    \end{column}
    \begin{column}{0.5\textwidth}
      \begin{itemize}
        \item code reuse
        \item code injection
        \item shellcode
        \item Return Oriented Programming (ROP)
        \item Data Oriented Programming (DOP)
        \item control flow
        \item control flow hijacking
        \item control flow integrity
      \end{itemize}
    \end{column}
  \end{columns}
\end{frame}

%\begin{frame}{Nice to read}
%  \begin{itemize}
%    \item \href{https://www-s.acm.illinois.edu/sigpony/old/talks/general\_exploitation/arc\_injection/arc\_injection.html}
%  \end{itemize}
%\end{frame}

\end{document}
