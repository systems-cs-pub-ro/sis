\documentclass{curs}

% Comment out lines below in case of no code to be included.
%\usepackage{code/highlight}
\usepackage{color}
\usepackage{graphicx}
%\usepackage{alltt}


\title[Session 05]{Session 05}
\subtitle{Defense and Mitigation}
\author{Security of Information Systems (SIS)}
\date{October 25, 2019}

\begin{document}

\frame{\titlepage}

\begin{frame}{Attack and Defense}
  \begin{itemize}
    \item attack: exploit vulnerabilities
    \item defense: prevent attacks, make attacks difficult, confine attacks
    \item attacker needs to find one security hole
    \item defender has to protect all security holes
    \item attacker invests time
    \item defense mechanisms incur overhead
  \end{itemize}
\end{frame}


\section{Attacker Perspective}

\begin{frame}{Attacker Goals}
  \begin{itemize}
    \item control
    \item cripple
    \item steal
  \end{itemize}
\end{frame}

\begin{frame}{Exploit}
  \begin{itemize}
    \item determine entries/input
    \item graph/automaton describes system/application behavior
    \item subvert graph
      \begin{itemize}
        \item add new nodes (inject)
        \item add new edges
        \item use existing paths in a different way
      \end{itemize}
  \end{itemize}
\end{frame}

\begin{frame}{Attack Vector}
  \begin{itemize}
    \item chain together multiple exploits
    \item gain access, gain privileged access, cripple, steal
    \item use vulnerabilities in software, system, web
  \end{itemize}
\end{frame}


\section{Defender Perspective}

\begin{frame}{Time}
  \begin{itemize}
    \item always on the attacker side
    \item prevent attacks is better than handling attacks
  \end{itemize}
\end{frame}

\begin{frame}{System Components}
  \begin{itemize}
    \item protect everything
    \item attacker need only find \textbf{one} flaw
    \item defense in depth
  \end{itemize}
\end{frame}

\begin{frame}{Prevention}
  \begin{itemize}
    \item preventive/proactive is better than reactive
    \item harden system components
    \item monitor everything
  \end{itemize}
\end{frame}

\begin{frame}{Security vs Speed}
  \begin{itemize}
    \item any defense mechanism incurs overhead
    \item use both offline (check at development time) and online mechanisms (check/harden during run-time)
  \end{itemize}
\end{frame}

\begin{frame}{Handling Complexity}
  \begin{itemize}
    \item automate processes
    \item verification and validation
    \item check before deployment
    \item prioritize critical parts
  \end{itemize}
\end{frame}

\begin{frame}{Monitoring}
  \begin{itemize}
    \item paranoia is a virtue
    \item frequent updates
    \item be on the lookout for CVEs
  \end{itemize}
\end{frame}


\section{Generic Defensive Techniques}

\begin{frame}{Defensive Steps}
  \begin{itemize}
    \item prevent existence and prevent exploitation
    \item during development
    \item before deployment
    \item during deployment: prevent, react, confine
  \end{itemize}
\end{frame}

\begin{frame}{Prevent Existence}
  \begin{itemize}
    \item prevent existence of bugs and vulnerabilities
    \item during development and before deployment
    \item Secure Software Development
    \item secure coding, defensive programming
    \item code auditing, code linting
    \item fuzzing, symbolic execution
  \end{itemize}
\end{frame}

\begin{frame}{Prevent Exploitation}
  \begin{itemize}
    \item during deployment
    \item if vulnerabilities exist, you cannot exploit them
    \item either prevent or make it harder for the attacker
    \item harden the application, the system
  \end{itemize}
\end{frame}

\begin{frame}{Making Exploitation it Harder}
  \begin{itemize}
    \item randomize
    \item obfuscate
    \item break application into multiple apps
    \item reduce number of inputs (attack surface)
  \end{itemize}
\end{frame}

\begin{frame}{Preventing Exploitation}
  \begin{itemize}
    \item make memory areas inaccessible
    \item isolate components
    \item harden executable with checker and sanitizers during runtime
    \item disadvantage: incurs overhead
  \end{itemize}
\end{frame}

\begin{frame}{Confine}
  \begin{itemize}
    \item more in session 7: Application Confinement
    \item \textbf{when} the attack happens, reduce damage
    \item sandboxing, permissions
    \item treat application as potential malware
  \end{itemize}
\end{frame}

\begin{frame}{React}
  \begin{itemize}
    \item monitor applications, system
    \item when attack happens, document, make app/system inaccessible
    \item patch as soon as possible
    \item investigate, prevent future similar attacks
  \end{itemize}
\end{frame}


\section{Application Defense}

\begin{frame}{Mindset}
  \begin{itemize}
    \item application is target of attacker
    \item input minimization, input validation
    \item you deploy an app that may have flaws or may be malware
    \item memory disclosure attacks, application control
  \end{itemize}
\end{frame}

\begin{frame}{Goal}
  \begin{itemize}
    \item prevent control flow hijacking
    \item prevent memory/information disclosure
    \item be on the look for policy flaws that may allow the app to leak information
  \end{itemize}
\end{frame}

\begin{frame}{CFI}
  \begin{itemize}
    \item \textit{Control Flow Integrity}
    \item make sure control flow graph is unchanged during run
    \item high overhead
    \item fine-grained vs coarse grained CFI
  \end{itemize}
\end{frame}

\begin{frame}{Code Pointers}
  \begin{itemize}
    \item critical memory data
    \item target for attacker for control flow hijacking
    \item function return addresses, function pointers
    \item \textit{Code Pointer Integrity} (faster approache to CFI), next lecture
  \end{itemize}
\end{frame}

\begin{frame}{Prevent Vulnerabilities vs Prevent Exploiting vs Make Unlikely vs Confine}
  \begin{itemize}
    \item prevent vulnerabilities: secure coding, verification, fuzzing, symbolic execution, type safety, safe programming languages (later sessions)
    \item prevent exploiting: ASan, StackGuard (canaries), SafeStack, CFI, input validation, DEP
    \item make unlikely: ASLR, multiple heaps
    \item confine: sandboxing, privacy settings, access control settings, SFI (\textit{Software Fault Isolation}) (later sessions)
  \end{itemize}
\end{frame}

\begin{frame}{Stack Guard / Address Sanitizer}
  \begin{itemize}
    \item stack canary, stack protector
    \item added at compile time
    \item value (canary) placed between buffer and return address
    \item overwriting canary is detected and ends the program
    \item may leak canary and overwrite it with itself
    \item may overwrite other data (without overwriting canary)
    \item may overwrite stack guard exit handler
    \item Google Address Sanitizer adds multiple checks, albeit at increased overhead
  \end{itemize}
\end{frame}

\begin{frame}{Input Validation}
  \begin{itemize}
    \item assume input is ``evil''
    \item prevent injection: command injection, SQL injection, shellcode injection
    \item prevent attacks such as billion laughs attacks
    \item prevent certain patterns, parse input
  \end{itemize}
\end{frame}

\begin{frame}{CFI}
  \begin{itemize}
    \item monitor control graph
    \item monitor calls, jumps, branches
    \item aim to do it without incurring significant overhead
    \item may happen offline
  \end{itemize}
\end{frame}

\begin{frame}{SafeStack}
  \begin{itemize}
    \item store code pointers in a separate stack
    \item buffer overflows will not overflow code pointers
    \item provide specific methods to access safe stack data
  \end{itemize}
\end{frame}

\begin{frame}{DEP}
  \begin{itemize}
    \item Data Execution Prevention
    \item mark writable memory area as non-executable
    \item you cannot write and execute, i.e. inject code
    \item data, heap, stack are marked with DEP
    \item may be bypassed by using a \texttt{mprotect()}-like call to update memory area permissions
  \end{itemize}
\end{frame}

\begin{frame}{ASLR}
  \begin{itemize}
    \item Address Space Layout Randomization
    \item new memory sections (especially libraries) are loaded at random addresses
    \item makes it difficult to find addresses
    \item not that effective on i386; useful on x86\_64
    \item may be bypassed by information leaking
  \end{itemize}
\end{frame}

\section{Web Application Defense}

\begin{frame}{General}
  \begin{itemize}
    \item secure configuration
    \item input sanitization
    \item trusted connection
    \item no vulnerable dependencies
  \end{itemize}
\end{frame}

\begin{frame}{Verification}
  \begin{itemize}
    \item client side
    \item server side
  \end{itemize}
\end{frame}

\begin{frame}{Connection}
  \begin{itemize}
    \item HTTPS, SSL/TLS
    \item certificate
    \item downgrade attacks
  \end{itemize}
\end{frame}

\begin{frame}{Secure HTTP Headers}
  \begin{itemize}
    \item HTTP Strict Transport Security (HSTS)
    \item X-Frame-Options
    \item X-XSS-Protection
    \item X-Content-Type-Options
    \item Content-Security-Policy
    \item Referrer-Policy
    \item Expect-CT
  \end{itemize}
\end{frame}

\begin{frame}{Database protection}
  \begin{itemize}
    \item \textbf{sanitize} queries
    \item encrypt data at rest
    \item encrypt data in transit
    \item \textbf{sanitize} queries
  \end{itemize}
\end{frame}


\section{Operating System Defense}

\begin{frame}{General System Defense}
  \begin{itemize}
    \item Intrusion Detection System
    \item Intrusion Prevention System
  \end{itemize}
\end{frame}

\begin{frame}{Signing}
  \begin{itemize}
    \item secure boot
    \item application signing
  \end{itemize}
\end{frame}

\begin{frame}{Sandboxing}
  \begin{itemize}
    \item Mandatory Access Control
      \begin{itemize}
        \item SELinux
        \item SMACK
        \item AppArmor
        \item TOMOYO
      \end{itemize}
    \item seccomp
  \end{itemize}
\end{frame}

\begin{frame}{Kernel Config}
  \begin{itemize}
    \item CONFIG_HARDENED_USERCOPY
    \item CONFIG_FORTIFY_SOURCE
    \item CONFIG_RANDOMIZE_BASE (KASLR)
    \item CONFIG_KASAN
    \item CONFIG_UBSAN
    \item In development
      \begin{itemize}
        \item KTSAN
        \item KMSAN
      \end{itemize}
    \item grsecurity
  \end{itemize}
\end{frame}


\section{Summary}

\begin{frame}{Defensive Mechanisms}
  \begin{itemize}
    \item prevent existence, prevent exploitation
    \item development, before deployment, during deployment
    \item input is the root of all evil
    \item look out for control flow hijacks, information leaks, malformed input
  \end{itemize}
\end{frame}

\begin{frame}{Keywords}
  \begin{columns}
    \begin{column}{0.5\textwidth}
      \begin{itemize}
        \item vulnerability
        \item exploit
        \item attack vector
        \item prevention
        \item isolation
        \item CFI
        \item code pointer
        \item Stack Guard
      \end{itemize}
    \end{column}
    \begin{column}{0.5\textwidth}
      \begin{itemize}
        \item DEP
        \item ASLR
        \item Address Sanitizer
        \item downgrade attacks
        \item secure HTTP headers
        \item sandboxing
        \item Mandatory Access Control
      \end{itemize}
    \end{column}
  \end{columns}
\end{frame}

\begin{frame}{Resources}
  \begin{itemize}
    \item \href{https://letsencrypt.org/}{Let's Encrypt}
    \item \href{https://www.youtube.com/watch?v=MFol6IMbZ7Y}{Defeating SSL Using Sslstrip}
    \item \href{https://www.owasp.org/index.php/OWASP_Secure_Headers_Project}{OWASP Secure Headers Project}
  \end{itemize}
\end{frame}

\end{document}
